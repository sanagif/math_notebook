\documentclass[a4paper,dvipdfmx]{jsarticle}

%基本パッケージ
\usepackage{amsmath,amssymb}
\usepackage{bm}
\usepackage{graphicx}
\usepackage{ascmac}
\usepackage{amsthm}

%定理等を資格で囲む
\usepackage{tcolorbox}
\tcbuselibrary{breakable, skins, theorems}

%脚注の形式を定義
\renewcommand\thefootnote{\arabic{footnote})}

\setlength{\textwidth}{\fullwidth}
\setlength{\textheight}{39\baselineskip}
\addtolength{\textheight}{\topskip}
\setlength{\voffset}{-0.5in}
\setlength{\headsep}{0.3in}


\theoremstyle{definition}
\newtheorem{theorem}{定理}
\newtheorem*{theorem*}{定理}
\newtheorem{definition}[theorem]{定義}
\newtheorem*{definition*}{定義}

%独自ショートカット
\newcommand{\Bee}{\mathfrak{B}}
\newcommand{\R}{\mathbb{R}}
\newcommand{\RR}{\overline{\mathbb{R}}}

\pagestyle{myheadings}
\markright{\footnotesize \sf sanagifの勉強ノート \ \texttt{https://github.com/sanagif/math\_notebook}}

\title{ルベーグ積分まとめ(執筆中)}
\author{sanagif}
\date{\today}

\begin{document}

\maketitle

\section*{はじめに}
伊藤清三先生の「Lebesgue積分入門(新装版)」で個人的に勉強したことをまとめました.
集合に関する基礎的な知識のみを前提として,Lebesugue積分の定義までを説明します.
ご指摘がある場合は,Githubアカウントの@sanagifまでご連絡ください.

\section{記号の定義}
集合$X$の部分集合$E$に対して,定義関数を$\chi_{E}$で表す.すなわち,
\begin{align}
    \chi_{E}(x) =
    \begin{cases}
        1 & (x \in E) \\
        0 & (x \in E^c)
    \end{cases}\notag
\end{align}.
\par

実数全体の集合$\R$に$+\infty,-\infty$を付加したものを$\RR$と表す.
すなわち$\RR = \R \cup \{+\infty, -\infty\}$.
\section{基礎: 可測空間,可測関数}
\begin{tcolorbox}
\begin{definition}
    集合$X$の部分集合からなる集合族$\Bee$が\textbf{$\sigma$-加法族}または\textbf{完全加法族}であるとは次の条件を満たすことである.
\begin{align}
    &\phi \in \Bee \\
    &E \in \Bee ならば E^c \in \Bee \\
    &E_n \in \Bee (n = 1, 2, \cdots) ならば \bigcup_{n=1}^{\infty} E_n \in \Bee
\end{align}
集合$X$と$X$上の加法族$\Bee$の組$(X,\Bee)$を\textbf{可測空間}と呼ぶ.
\end{definition}
\end{tcolorbox}

\par

可測空間$(X,\Bee)$を考える.
1つの集合$E \in \Bee$を固定する.
写像$f: E \to \R \cup \{+\infty, -\infty\}$と実数$a$に対して次のような記法を採用する.
\begin{align}
    E(f < a) = \{ x \in E \mid f(x) < a\}
\end{align}
同様に
\begin{align}
    E(f \le a) = \{ x \in E \mid f(x) \le a\} \\
    E(f = a) = \{ x \in E \mid f(x) = a\} \\
    E(a < f \le b) = \{ x \in E \mid a < f(x) \le a\}
\end{align}
などと定義する.

\begin{tcolorbox}
\begin{definition}
    可測空間$(X,\Bee)$と1つの集合$E \in \Bee$に対して,
    写像$f: E \to \R \cup \{+\infty, -\infty\}$が\textbf{可測関数}であるとは,任意の実数$a$に対して,
    $E(f < a) \in \Bee$ \footnotemark を満たすことである.
\end{definition}
\end{tcolorbox}

\begin{tcolorbox}
\begin{definition}
可測空間$(X,\Bee)$と1つの集合$E \in \Bee$に対して,
    写像$f: E \to \R \cup \{+\infty, -\infty\}$が\textbf{単関数}あるいは\textbf{階段関数}であるとは,
    集合$E$が$E = E_1 + E_2 + \cdots + E_n$と有限個の直和に表され,任意の$x \in E$に対して
    \begin{align*}
        f(x) = \sum_{i=1}^{\infty}\alpha_{i}\chi_{E_i}(x)
    \end{align*}
    が成立することを言う.
    ただし,各$\alpha_{i}$は実数で$i \neq j$ならば$\alpha_{i} \neq \alpha_{j}$.
\end{definition}
\end{tcolorbox}

\footnotetext{{$E(f \le a) \in \Bee, E(f > a) \in \Bee, E(f \ge a) \in \Bee$に置き換えても良い}}
\end{document}
